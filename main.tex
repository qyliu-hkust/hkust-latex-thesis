%%%%%%%%%%%%%%%%%%%%%%%%%%%%%%%%%%%%%%%%%%%%%%%%%%%%%%%%%%%%%%%%%%%%%%%%%
%                                                                       %
% ustthesis_test.tex: A template file for usage with ustthesis.cls      %
%                                                                       %
%%%%%%%%%%%%%%%%%%%%%%%%%%%%%%%%%%%%%%%%%%%%%%%%%%%%%%%%%%%%%%%%%%%%%%%%%

\documentclass{ustthesis}

\usepackage{mathpazo,amsmath,amssymb,epsfig,enumerate,bbm,calc,color,ifthen,capt-of} % original was times, but I think it's ugly; we use the same as IEEE CompSoc
\usepackage{algorithm}
\usepackage[noend]{algorithmic}
\usepackage[center]{subfigure}
\usepackage{color,graphicx}
\newtheorem{proof}{Proof}
\usepackage{hyperref} % for better viewing experience  -- added by alan
\usepackage[margin=25mm,textheight=247mm,textwidth=145mm]{geometry}

% Alan: begin the font trial
% Euler for math | Palatino for rm | Helvetica for ss | Courier for tt
\renewcommand{\rmdefault}{ppl} % rm
%\linespread{1.05}        % Palatino needs more leading
\usepackage[scaled]{helvet} % ss
\usepackage{courier} % tt
%\usepackage{euler} % math
\usepackage{eulervm} % a better implementation of the euler package (not in gwTeX)
\normalfont
\usepackage[T1]{fontenc}
% Alan: end the font trial

\newcommand{\red}[1]{#1}
\newcommand{\tab}[1]{\hspace{3mm}}

% \usepackage{latexsym}
    % Use the "latexsym" package when encountering the following error:
    %   ! LaTeX Error: Command \??? not provided in base LaTeX2e.
% \usepackage{epsf}
    % Use the "epsf" package for including EPS files.

%%%%%%%%%%%%%%%%%%%%%%%%%%%%%%%%%%%%%%%%%%%%%%%%%%%%%%%%%%%%%%%%%%%%%%%%%
%                                                                       %
% Preambles. DO NOT ERASE THEM. Change to suite your particular purpose.%
%                                                                       %
%%%%%%%%%%%%%%%%%%%%%%%%%%%%%%%%%%%%%%%%%%%%%%%%%%%%%%%%%%%%%%%%%%%%%%%%%

\title{Towards Maximum Points Enclosing Problems and Related Spatial Query Processing}  % Title of the thesis.
\author{Qiyu~LIU}     % Author of the thesis.             
% Degree for which the thesis is.
%\degree{\mphil} or
%\degree{\PhD}              % Degree for which the thesis is.
\department{Computer Science and Engineering}       % Department to which the thesis
                    % is submitted.
\advisor{Lei~Chen}     % Supervisor.
% \depthead{Prof.~Mounir~HAMDI}    % department head.
\defencedate{2019}{08}{19}      % \defencedate{year}{month}{day}.

% NOTE:
%   According to the sample shown in the guidelines, page number is
%   placed below the bottom margin.  However, if the author prefers
%   the page number to be printed above the bottom margin, please
%   activate the following command.

% \PNumberAboveBottomMargin

\begin{document}

%%%%%%%%%%%%%%%%%%%%%%%%%%%%%%%%%%%%%%%%%%%%%%%%%%%%%%%%%%%%%%%%%%%%%%%%%
%                                                                       %
% Now the actual Thesis. The order of output MUST be followed:          %
%                                                                       %
%    1) TITLEPAGE                                                       %
%                                                                       %
% The \maketitle command generates the Title page as well as the        %
% Signature page.                                                       %
%                                                                       %
%%%%%%%%%%%%%%%%%%%%%%%%%%%%%%%%%%%%%%%%%%%%%%%%%%%%%%%%%%%%%%%%%%%%%%%%%

\maketitle

%%%%%%%%%%%%%%%%%%%%%%%%%%%%%%%%%%%%%%%%%%%%%%%%%%%%%%%%%%%%%%%%%%%%%%%%%
%                                                                       %
%     2) DEDICATION (Optional)                                          %
%                                                                       %
% The \dedication and \enddedication commands are optional. If          %
% specified it generates a page for dedication.                         %
%
%%%%%%%%%%%%%%%%%%%%%%%%%%%%%%%%%%%%%%%%%%%%%%%%%%%%%%%%%%%%%%%%%%%%%%%%%

% \dedication
% This is an optional section.
% \enddedication

%%%%%%%%%%%%%%%%%%%%%%%%%%%%%%%%%%%%%%%%%%%%%%%%%%%%%%%%%%%%%%%%%%%%%%%%%
%                                                                       %
%     3) ACKNOWLEDGMENTS                                                %
%                                                                       %
% \acknowledgments and \endacknowledgments defines the                  %
% Acknowledgments of the author of the Thesis.                          %
%                                                                       %
%%%%%%%%%%%%%%%%%%%%%%%%%%%%%%%%%%%%%%%%%%%%%%%%%%%%%%%%%%%%%%%%%%%%%%%%%

% \input{chapter/sec-ack}
%%%%%%%%%%%%%%%%%%%%%%%%%%%%%%%%%%%%%%%%%%%%%%%%%%%%%%%%%%%%%%%%%%%%%%%%%
%                                                                       %
%     4) TABLE OF CONTENTS                                              %
%                                                                       %
%%%%%%%%%%%%%%%%%%%%%%%%%%%%%%%%%%%%%%%%%%%%%%%%%%%%%%%%%%%%%%%%%%%%%%%%%

\tableofcontents

%%%%%%%%%%%%%%%%%%%%%%%%%%%%%%%%%%%%%%%%%%%%%%%%%%%%%%%%%%%%%%%%%%%%%%%%%
%                                                                       %
%     5) LIST OF FIGURES (If Any)                                       %
%                                                                       %
%%%%%%%%%%%%%%%%%%%%%%%%%%%%%%%%%%%%%%%%%%%%%%%%%%%%%%%%%%%%%%%%%%%%%%%%%

% \listoffigures

%%%%%%%%%%%%%%%%%%%%%%%%%%%%%%%%%%%%%%%%%%%%%%%%%%%%%%%%%%%%%%%%%%%%%%%%%
%                                                                       %
%     6) LIST OF TABLES (If Any)
%                                                                       %
%%%%%%%%%%%%%%%%%%%%%%%%%%%%%%%%%%%%%%%%%%%%%%%%%%%%%%%%%%%%%%%%%%%%%%%%%

% \listoftables

%%%%%%%%%%%%%%%%%%%%%%%%%%%%%%%%%%%%%%%%%%%%%%%%%%%%%%%%%%%%%%%%%%%%%%%%%
%                                                                       %
%     7) ABSTRACT                                                       %
%                                                                       %
% \abstract and \endabstract are used to define a short Abstract for    %
% the Thesis.                                                           %
%                                                                       %
%%%%%%%%%%%%%%%%%%%%%%%%%%%%%%%%%%%%%%%%%%%%%%%%%%%%%%%%%%%%%%%%%%%%%%%%%

\begin{abstract}
As a classical computational geometry problem, the Maximum Points Enclosing problem (MPE) has a wide spectrum of utilities, and can be regarded as building blocks of solutions to many real-world applications like region of interest (ROI) detection, bust. 
\end{abstract}

\chapter{Introduction}\label{chapter:introduction}
\section{Something}

\section{Something1}

\bibliographystyle{plain}

% \bibliography{ref}

%%%%%%%%%%%%%%%%%%%%%%%%%%%%%%%%%%%%%%%%%%%%%%%%%%%%%%%%%%%%%%%%%%%%%%%%%
%                                                                       %
%     8) The Actual Contents                                            %
%                                                                       %
% The command \chapters MUST BE USED to ensure that the entire content  %
% of the Thesis is double-spaced (in version 1.0).                      %
%                                                                       %
% However, in version 2.0, \chapters will be automatically added in     %
% the beginning of the first chapter.                                   %
%                                                                       %
%%%%%%%%%%%%%%%%%%%%%%%%%%%%%%%%%%%%%%%%%%%%%%%%%%%%%%%%%%%%%%%%%%%%%%%%%

%%\chapters         % Not necessary with ustthesis.cls (v2.0).

%%%%%%%%%%%%%%%%%%%%%%%%%%%%%%%%%%%%%%%%%%%%%%%%%%%%%%%%%%%%%%%%%%%%%%%%%
%                                                                       %
% Each chapter is defined via the \chapter command. The usual sectional %
% commands of LaTeX are also available.                                 %
%                                                                       %
%%%%%%%%%%%%%%%%%%%%%%%%%%%%%%%%%%%%%%%%%%%%%%%%%%%%%%%%%%%%%%%%%%%%%%%%%


%%%%%%%%%%%%%%%%%%%%%%%%%%%%%%%%%%%%%%%%%%%%%%%%%%%%%%%%%%%%%%%%%%%%%%%%%
%                                                                       %
%      9) BIBLIOGRAPHY                                                  %
%                                                                       %
% This example uses bibtex to generate the required Bibliography. Refer %
% to the % the file ustthesis_test.bib for the entries of the           %
% Bibliography. Note that only the cited entries are printed.           %
%                                                                       %
% If BibTeX is not used to typeset the bibliography, replace the        %
% following line with the \begin{thebibliography} and \end{bibliography}%
% commands (the "thebibliography" environment) to process the           %
% Bibliography.                                                         %
%                                                                       %
%%%%%%%%%%%%%%%%%%%%%%%%%%%%%%%%%%%%%%%%%%%%%%%%%%%%%%%%%%%%%%%%%%%%%%%%%

%%%%%%%%%%%%%%%%%%%%%%%%%%%%%%%%%%%%%%%%%%%%%%%%%%%%%%%%%%%%%%%%%%%%%%%%%
%                                                                       %
% The recommended bibliography style is the IEEE bibliography style.    %
% "ustbib" defines the IEEE bibliography standard with the added        %
% ability of sorting the items by name of author.                       %
%                                                                       %
% If you are not using BibTeX to process your Bibliography, comment out %
% the following line.                                                   %
%                                                                       %
%%%%%%%%%%%%%%%%%%%%%%%%%%%%%%%%%%%%%%%%%%%%%%%%%%%%%%%%%%%%%%%%%%%%%%%%%




% Please run "bibtex ustthesis_test" before the bibliography can be
% included.

%%%%%%%%%%%%%%%%%%%%%%%%%%%%%%%%%%%%%%%%%%%%%%%%%%%%%%%%%%%%%%%%%%%%%%%%%
%                                                                       %
%     10) APPENDIX (If Any)                                              %
%                                                                       %
% \appendix command marks the beginning of the APPENDIX part of the     %
% Thesis. The usual \chapter command is used for the different chapters %
% of the Appendix.                                                      %
%                                                                       %
%%%%%%%%%%%%%%%%%%%%%%%%%%%%%%%%%%%%%%%%%%%%%%%%%%%%%%%%%%%%%%%%%%%%%%%%%


%%%%%%%%%%%%%%%%%%%%%%%%%%%%%%%%%%%%%%%%%%%%%%%%%%%%%%%%%%%%%%%%%%%%%%%%%
%                                                                       %
%     11) BIOGRAPHY (Optional)                                          %
%                                                                       %
% \biography and \endbiography are used to define the optional          %
% Biography of the author of the Thesis.                                %
%                                                                       %
%%%%%%%%%%%%%%%%%%%%%%%%%%%%%%%%%%%%%%%%%%%%%%%%%%%%%%%%%%%%%%%%%%%%%%%%%

% \biography
% The biography of the student is ALSO optional.
% \endbiography

\end{document}
